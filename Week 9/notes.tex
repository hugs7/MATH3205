Markowitz or Mean/Variance Models

Sets
$N$ assets we want to invest in

Data
$r_i$ return on asset $i \in N$
Typically these are slightly greater than 1.

$W$ Covariance Matrix

We assume the returns have a multivariate normal distribution

Variables

$x_i$ is the fraction of our portfolio investged in each asset

Maximise return

\[\max (\lambda sum_{i\in N} r_i x_i - (1-\lambda) x^T w X)\]

with
$\lambda \in (0,1)$

subject to

\[\sum_{i\in N} x_i = 1\]

---

General second order cone constraint

\[\boldsymbol{x}^\top W \boldsymbol{x} \le z\]

Where $x\ge 0$ and $z \ge 0$ are variables and $W$ is a positive semi-definite matrix.



Gurobi can do second order cone constraints because the Barrier algorithm turns regular constraints into these.
Solvable in polynomial time

---

Value at Risk and Conditional Value at Risk
(VaR) (CVaR)

Sample average approximation (SAA)

Assume we want VaR and CVaR at the $\alpha$ level (e.g.\ the bottom $5\%$).

Sets
$S$ samples

Data
$r_{is}$ return for investment $i \in N$ in sample $s \in S$

Variables
$\beta_s$ return of sample $s \in S$.
$\text{Var}$ estimate of VaR from samples
$\text{CVar}$ esstimate of CVaR from samples
$\beta^{-}$ the amount by which the return of scenario $s \in S$ is below $\text{Var}$

Objective

\[ \max \frac{\lambda}{|S|} \sum_{s \in S} + (1 - \lambda)\text{CVar} \]

Subject to

\[ \sum_{i\in N} x_i = 1 \]

\[ \beta_s = \sum_{ i \in N } r_{is} X_i, \hspace{1cm} \forall s \in S \]

\[ \beta_s + \beta_s^{-} \ge \text{Var}, \hspace{1cm} \forall s \ in S \]

\[ \text{CVar} = \text{Var} - \frac{1}{\alpha |S|} \sum_{s \in S} \beta^{-} \]

---

Chance constrained portfolio Models
\[ \max \sum_{i \in N} r_i x_i \]

Subject to

\[ \sum_{i \in N} = 1 \]

Chance constraint for small values of $\alpha$

\[ pr\left(\sum_{i \in N} r_i x_i \ge 1\right) \ge (1 - \alpha) \]

This turns into 
\[ \left(F^{-1} \left(1 - \alpha \right)\right)^2 x^\top W x \le \left(1 - \sum_{i \ in N} r_i x_i \right)^2 \]


What if the data is not normally distribted

We can generate a lot of samples.
Use Campi & Garatti formula to calculate $k$: the numnber of failures permitted for a given sample size
$\alpha$ and degrees of freedom. Replace $W$ with the sample variance and $r$ with the sample average returns. 

Treat $F^{-1}(1-\alpha)$ as a parameter above, and repeatedly solve and count the number of failures, 
aiming to get exactly $k$ failures (via line search/interpolation techniques).

