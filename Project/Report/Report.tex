\documentclass{article}

\usepackage{amsmath, amssymb, amsthm}
\usepackage[margin=1in]{geometry}
\usepackage{tcolorbox}
\usepackage{fancyvrb}
\usepackage[hidelinks]{hyperref}
\usepackage{tikz, wasysym, tikz-cd}
\usepackage{mathtools} %xrightarrow
\usetikzlibrary{automata,arrows.meta,positioning,calc,decorations.pathmorphing}
\usepackage{etoolbox} % setcounter{counter}{num}
\usepackage{enumitem} % \begin{enumerate}[label=(\alph*)]

% QUIVER STUFF
% A TikZ style for curved arrows of a fixed height, due to AndréC.
\tikzset{curve/.style={settings={#1},to path={(\tikztostart)
    .. controls ($(\tikztostart)!\pv{pos}!(\tikztotarget)!\pv{height}!270:(\tikztotarget)$)
    and ($(\tikztostart)!1-\pv{pos}!(\tikztotarget)!\pv{height}!270:(\tikztotarget)$)
    .. (\tikztotarget)\tikztonodes}},
    settings/.code={\tikzset{quiver/.cd,#1}
        \def\pv##1{\pgfkeysvalueof{/tikz/quiver/##1}}},
    quiver/.cd,pos/.initial=0.35,height/.initial=0}
% TikZ arrowhead/tail styles.
\tikzset{tail reversed/.code={\pgfsetarrowsstart{tikzcd to}}}
\tikzset{2tail/.code={\pgfsetarrowsstart{Implies[reversed]}}}
\tikzset{2tail reversed/.code={\pgfsetarrowsstart{Implies}}}
% TikZ arrow styles.
\tikzset{no body/.style={/tikz/dash pattern=on 0 off 1mm}}

\theoremstyle{plain}
\newtheorem{theorem}{Theorem}
\numberwithin{theorem}{section}
\newtheorem{lemma}{Lemma}
%\numberwithin{lemma}{section}
\newtheorem{example}{Example}
\numberwithin{example}{section}

\theoremstyle{definition}
\newtheorem{definition}{Definition}
%\numberwithin{definition}{section}
\newtheorem{problem}{Problem}
%\numberwithin{problem}{section}

\newenvironment{solution}
  {\renewcommand\qedsymbol{$\blacksquare$}\begin{proof}[Solution]}
  {\end{proof}}

\newtcolorbox{codebox}[1][] {
    colback = black!2!white,
    colframe = black,
    sharp corners,
    title = \texttt{#1}
}

\newtcolorbox{problembox} {
    colback = white,
    colframe = black,
    sharp corners
}

\newtcolorbox{defbox} {
    colback = blue!5!white,
    colframe = blue!75!black,
    sharp corners
}

\newtcolorbox{lemmabox} {
    colback = green!5!white,
    colframe = green!75!black,
    sharp corners
}

\newtcolorbox{theorembox} {
    colback = red!5!white,
    colframe = red!75!black,
    sharp corners
}

\newcommand{\Sup}{\text{Sup}}
\newcommand{\Inf}{\text{Inf}}
\DeclareMathOperator{\im}{im}
\DeclareMathOperator{\dom}{dom}
\DeclareMathOperator{\cod}{cod}
\DeclareMathOperator{\id}{id}
\newcommand{\angles}[1]{\left<#1\right>}
\newcommand{\catname}[1]{{\normalfont\textbf{#1}}}
\newcommand{\GL}{\text{GL}}
\newcommand{\SL}{\text{SL}}
\newcommand{\sg}{\text{sg}}
\newcommand{\dd}{\mathop{}\!d}
\newcommand{\x}{{\bf x}}
\newcommand{\y}{{\bf y}}
\newcommand\defeq{\stackrel{\mathclap{\normalfont\mbox{\scriptsize def}}}{=}}

\DeclarePairedDelimiterX\set[1]\lbrace\rbrace{\def\given{\;\delimsize\vert\;}#1}
 
\newcommand{\naturalto}{%
  \mathrel{\vbox{\offinterlineskip
    \mathsurround=0pt
    \ialign{\hfil##\hfil\cr
      \normalfont\scalebox{1.2}{.}\cr
%      \noalign{\kern-.05ex}
      $\longrightarrow$\cr}
  }}%
}

% Actions to be done at the start of a new section, so just made a new macro
% \newcommand{\Section}[1]{\section{#1} \setcounter{equation}{0}}

\setlength{\parindent}{0pt}

\title{MATH3205 - Report}
\author{Hugo Burton, Anna Henderson and Mitchell Holt}

\begin{document}

\maketitle

\tableofcontents

\newpage

\section{Benders Formulation}
\subsection{Master Problem (BMP)}
The master problem will assign events to time periods (with no consideration of
how events will be assigned to rooms within each time period).

\subsubsection*{Sets}
\begin{itemize}
    \item $E$ the set of events
    \item $P$ the set of periods
    \item $PA_e \subseteq P$ the set of available periods for event $e \in E$
    \item $PD \subseteq E$ the set of all events with period preferences.
    \item $PU \subseteq E$ the set of all events with undesired periods.
    \item $PB \subseteq P$ the set of all periods which are undesired (for all
        events).
    \item $P1$ the set of all unordered pairs $s$ of events $s = \{e_1, e_2\}$
        for which there exists a curriculum with courses $c_1 \neq c_2$, not
        both primary courses, such that $e_1$ and $e_2$ are examination events
        for $c_1$ and $c_2$ respectively.
    \item $P2 \subseteq E$ the set of all events which have undesired rooms.
    \item $P3$ the set of all unordered pairs $s$ of events $s = \{e_1, e_2\}$
        which satisfy any of:
        \begin{itemize}
            \item events from the same (written and oral) examination,
            \item events of \emph{consecutive} examinations from the same
                course, or
            \item there exists a curriculum with courses $c_1 \neq c_2$ such
            that $e_1$ and $e_2$ are examination events for $c_1$ and $c_2$
            respectively.
        \end{itemize}
    \item $C$ the set of curricula.
    \item $T$ the set of teachers.
\end{itemize}

\subsubsection*{Data}
\begin{itemize}
    \item $RA_e$ the set of rooms to which the event $e \in E$ could be
        assigned.
    \item $PA_e$ the set of periods to which the event $e \in E$ could be
        assigned.
    \item SOFT\_CONFLICT$_s$ the $S1$ penalty for the unordered event pair $s \in
        P1$.
    \item UNDESIRED\_PERIOD the fixed penalty for assigning an event $e \in PU$
        to an undesired period.
    \item NOT\_PREFERRED the fixed penalty for assigning an event $e \in PD$ to
        a period other than that which is preferred.
    \item UNDESIRED\_ROOM the fixed penalty for assigning an event $e \in P2$ to
        an undesired room.
    \item $UP_e$ the set of undesired periods for $e \in PU$.
    \item $PP_e$ the set of preferred periods for $e \in PD$.
    \item $UR_e \subset RA_e$ the set of undesired rooms for $e \in P2$.
    \item $R_p$ the set of rooms available in the period $p \in P$.
    \item $C^P_c$ the set of events belonging to primary courses of curriculum
        $c \in C$.
    \item $C^T_t$ the set of events belonging to courses taught by teacher $t
        \in T$.
\end{itemize}

\subsubsection*{Variables}
\begin{itemize}
    \item $y_{e, p} \in \{0,1\}$ the decision variable assigning events $e$ to
        periods $p$.
    \item $S1_s$ the penalty for a soft conflict between the events in the
        unordered pair events $s$, for all $s \in P1$.
    \item $S2_p$ the estimate of the number of events allocated to undesired
        rooms in period $p \in P$.
    \item $S3_s$ the estimate of the penalty for non-ideal distances between the
        events in the unordered pair $s$, for all $s \in P3$.
\end{itemize}

\subsubsection*{Objective}
\begin{multline}
    \min \quad \sum_{s \in P1} S1_s \quad + \quad \text{UNDESIRED\_PERIOD}
        \left( \sum_{p \in PB} \sum_{e \in E} y_{e, PB} +
        \sum_{e \in PU} \sum_{p \in UP_e} y_{e, p} \right) \\
    + \quad \text{NOT\_PREFERRED} \sum_{e \in PD} \sum_{p \in PP_e} (1 - y_{ep})
        \quad + \quad \text{UNDESIRED\_ROOM} \sum_{p \in P} S2_p \quad + \quad
        \sum_{s \in P3} S3_s
\end{multline}

\subsubsection*{Constraints}
Each event must be scheduled to exactly one time period (BSP (\S\ref{BSP}) will
allocate events to rooms once the period assignment has been fixed here):
\begin{equation}
    \sum_{p \in P} y_{e,p} = 1 \qquad \forall e \in E
\end{equation}

We need to make sure that $S1$ is exactly the penalty for a soft conflict (two
events from a common curriculum being scheduled at the same time). To do this,
we will ensure that $S1_s \geq 0$ (which can be implemented by setting the lower
bound for the variable in the Gurboi API) and that it achieves the cost if and
only if two events are in soft conflict:
\begin{gather}
    S1_s \geq 0 \qquad \forall s \in P1 \\
    S1_{\{e_1, e_2\}} \geq \text{SOFT\_CONFLICT}_{\{e_1, e_2\}}
        (y_{e_1, p} + y_{e_2, p} - 1)
        \qquad \forall \{e_1, e_2\} \in P1, \ \forall p \in P
\end{gather}

We can enforce that no two courses in H3-conflict are scheduled in the same time
period with the following constraints:
\begin{gather}
    \sum_{e \in C^P_c \; : \; p \in PA_e} y_{e,p} \leq 1
        \qquad \forall c \in C, \ \forall p \in P \\
    \sum_{e \in C^T_t \; : \; p \in PA_e} y_{e, p} \leq 1
        \qquad \forall t \in T \ \forall p \in P
\end{gather}

\subsubsection*{Lazy Constraints}
We write $(\cdot)^*$ to denote the value of a variable in some fixed solution to
BMP. For each period $p \in P$, let $E_p := \set{e \in E \given y^*_{e, p} =
1}$, the set of events allocated to period $p$, and do the following:
\begin{itemize}
    \item Solve BSP to obtain a room allocation for the events of $E_p$ to the
        rooms available in $p$ which minimizes the number of events allocated to
        undesired rooms. Let $RM_e$ be the room assigned to event $e \in E_p$.
    \item If BSP is infeasible, add feasibility cuts TODO.
    \item If BSP has a non-zero objective value $M$ (which is, by construction
        of BSP, the minimal number of undesired room allocations given $E_p$),
        we add the constraint:
        \begin{equation}
            S2_p \geq M \left(1 - \sum_{e \in E_p \cap P2} (1 - y_{e, p})\right)
        \end{equation}
        Indeed, we can add this constraint on $S2_{p'}$ for each $p' \in P$ with
        $R_{p'} \subseteq R_p$.
\end{itemize}
We also need to calculate the $S3$ penalties at the current solution and add
optimality cuts. TODO

\subsection{Sub-problem (BSP)} \label{BSP}
The sub-problem will allocate events to rooms within a fixed time period $p$
once they have been scheduled by the master problem, minimizing the number of
events assigned to an undesired room.

\subsubsection*{Sets}
\begin{itemize}
    \item $R^c \subseteq R$ the set of composite rooms.
    \item $R^0_{r^c}$ the set of rooms (irreducible or composite) overlapping
        with the composite room $r^c \in R^c$.
\end{itemize}

\subsubsection*{Variables}
Let $x_{e,r}$ be the decision variable assigning events $e \in E_p$ to rooms $r
\in RA_e \cap R_p$.

\subsubsection*{Objective}
\begin{equation}
    \min \qquad \sum_{e \in E_p \cap P2} \
        \sum_{r \in UR_e \cap R_p} x_{e,r}
\end{equation}

\subsubsection*{Constraints}
First we add constraints ensuring each event is allocated exactly one room, and
no two events are allocated the same room:
\begin{gather}
    \sum_{r \in RA_e \cap R_p} x_{e,r} = 1 \qquad \forall e \in E_p \\
    \sum_{e \in E_p \; : \; r \in RA_e} x_{e, r} \leq 1 \qquad \forall r \in R_p
\end{gather}

We can ensure that, whenever an event is allocated to a composite room, no event
is allocated to an overlapping room:
\begin{equation}
    |R^0_{r^c} \cap R_p| \sum_{e \in E_p} x_{e, r^c}
        + \sum_{r^0 \in R^0_{r^c} \cap R_p} \ \sum_{e \in E_p} x_{e, r^0}
        \leq |R^0_{r^c} \cap R_p| \qquad \forall r^c \in R^c \cap R_p
\end{equation}

\end{document}
